\documentclass[11pt,a4paper,spanish]{article}
\usepackage[latin1]{inputenc}
\usepackage{amsmath}
\usepackage{amsfonts}
\usepackage{amssymb}
\usepackage{graphicx}
\usepackage{multicol}
\usepackage{mathtools}
\DeclarePairedDelimiter\floor{\lfloor}{\rfloor}
\usepackage[boxruled, linesnumbered]{algorithm2e}
\usepackage[left=1.65cm, right=1.65cm, top=1.78cm, bottom=1.78cm]{geometry}

\author{Bollschweiler Vargas Ian Nicolas\\
			Cabrejos Rafael Erika Epifania\\
			Hidalgo Esquivel Jared Miguel\\
		 	Del Rosario Sosa Joshua Jean Paul	}
\title{\bf TIEMPO DE VALOR ESPERADO EN ALGORITMOS DE ORDENACI\'ON}
\date{}
\pagestyle{empty}
\begin{document}
	\maketitle
\begin{center}
		\textbf{Resumen}
	\begin{abstract}
	
	
	%contenido del resumen 
	\end{abstract}
\end{center}
\begin{multicols}{2}
\section{\normalsize INTRODUCCI\'ON}
\qquad Introducci�n de prueba
\section{{\normalsize ESTADO DEL ARTE}}
\begin{itemize}
\item \textbf{Shell Sort :} Es principalmente una variaci�n de Insertion Sort. En la ordenaci�n por inserci�n, movemos los elementos solo una posici�n m�s adelante. Cuando un elemento tiene que moverse mucho m�s adelante, muchos movimientos est�n involucrados.
Durante la ejecuci\'on de este algoritmo, los n\'umeros de la lista se van casi-ordenando y finalmente, el \'ultimo paso o funci\'on de este algoritmo es un simple m\'etodo por inserci\'on que, al estar casi-ordenados los n\'umeros, es m�s eficiente.
\item \textbf{Merge Sort:} El algoritmo de Ordenamiento por mezcla (Merge sort en ingl�s) es un algoritmo de ordenaci�n externo estable basado en la t�cnica divide y vencer�s. Es de complejidad O($n \log n$).Fue desarrollado en 1945 por John Von Neumann.
A grandes rasgos, el algoritmo consiste en dividir en dos partes iguales el vector a ordenar, ordenar por separado cada una de las partes, y luego mezclar ambas partes, manteniendo el orden, en un solo vector ordenado.
\item\textbf{Quick Sort:} El ordenamiento r�pido (quick sort en ingl�s) es un algoritmo basado en la t�cnica de divide y vencer�s, que permite, en promedio, ordenar n elementos en un tiempo proporcional a O($n \log n$). Esta es la t�cnica de ordenamiento m�s r�pida conocida. Fue desarrollada por C. Antony R. Hoare en 1960. El algoritmo original es recursivo, pero se utilizan versiones iterativas para mejorar su rendimiento.

\end{itemize}
\section{{\normalsize DISE�O DEL EXPERIMENTO}}
Para evaluar el tiempo de valor esperado, implementaremos 5 algoritmos de ordenamiento, siendo estos: Merge sort, Bubble sort, Quick sort, Insertion sort y Bucket sort. Estos ser�n implementados en lenguaje R donde ordenar�n un arreglo de n�meros aleatorios de n elementos. Este proceso se realizar� 100 veces y se tomar� la media de los tiempos de ejecuci�n de cada algoritmo para aproximar el tiempo de valor medio para esa cantidad de elementos. Despu�s, se proceder� a variar dicha cantidad para evaluar el crecimiento de la funci�n tiempo de ejecuci�n. Finalmente, se realizar� un ajuste para encontrar la funci�n que mejor describa el comportamiento de cada algoritmo para n elementos.\newline Los pseudoc�digos de los algoritmos a utilizar son:


\begin{algorithm}[H]	
	\SetKwData{A}{A}
    \SetKwFunction{Merge-Sort}{Merge-Sort}\SetKwFunction{Merge}{Merge}
    \SetKwInOut{Input}{Entrada}\SetKwInOut{Output}{Salida}
    \Input{Array A,p,r}
    \Output{Array A ordenado}
    \BlankLine
    \If(){p $<$ r}{
        \emph{ q = $\floor*{\frac{(p+r)}{2}}$}\;
        \Merge-Sort{ A,p,q}\;
        \Merge-Sort{ A,q+1,r}\;
        \Merge{A,p,q,r}\;
    }
    	\caption{Merge Sort}
\end{algorithm}


\begin{algorithm}[H]	
	\SetKwData{A}{A}
    \SetKwFunction{Merge-Sort}{Merge-Sort}\SetKwFunction{Merge}{Merge}
    \SetKwInOut{Input}{Entrada}\SetKwInOut{Output}{Salida}
    \Input{Array A con n elementos}
    \Output{Array A ordenado}
    \BlankLine
    \For{$i\leftarrow 1$ \KwTo $n$}{
		\For{$j\leftarrow 1$ \KwTo $n$}{        
        	\If(){A[j] $>$ A[j+1]}{
        		\emph{Intercambiar A[j] y A[j+1]}\;
			}        
        }
    }
    	\caption{Bubble Sort}
\end{algorithm}

\begin{algorithm}[H]	
	\SetKwData{A}{A}
    \SetKwFunction{Merge-Sort}{Merge-Sort}\SetKwFunction{Merge}{Merge}
    \SetKwInOut{Input}{Entrada}\SetKwInOut{Output}{Salida}
    \Input{Array A,p,r}
    \Output{Array A ordenado}
    \BlankLine
    \If(){p $<$ r}{
        \emph{ q = $\floor*{\frac{(p+r)}{2}}$}\;
        \Merge-Sort{ A,p,q}\;
        \Merge-Sort{ A,q+1,r}\;
        \Merge{A,p,q,r}\;
    }
    	\caption{Quick Sort}
\end{algorithm}

\begin{algorithm}[H]	
	
	\SetKwInOut{Input}{Entrada}\SetKwInOut{Output}{Salida}
    \Input{Array A con n elementos}
    \Output{Array A ordenado}
    \For{$j\leftarrow 2$ \KwTo n}{
		\emph{llave $=$ $A[j]$}\;
		\emph{$i = j-1$}\;
		\While(){i$>$0 y $A[i] > llave$}{
			\emph{$A[i+1] = A[i]$}\;
			\emph{$i = i-1$}\;
		}
		\emph{$A[i+1] = llave$}\;	
	}
    	\caption{Insertion Sort}
\end{algorithm}

\begin{algorithm}[H]
	\SetKwData{Left}{left}\SetKwData{This}{this}\SetKwData{Up}{up}
	\SetKwFunction{Union}{Union}\SetKwFunction{FindCompress}{FindCompress}						\SetKwInOut{Input}{input}\SetKwInOut{Output}{output}
	\Input{A bitmap $Im$ of size $w\times l$}
	\Output{A partition of the bitmap}
	\BlankLine
	\emph{special treatment of the first line}\;
	\For{$i\leftarrow 2$ \KwTo $l$}{
		\emph{special treatment of the first element of line $i$}\;
		\For{$j\leftarrow 2$ \KwTo $w$}{\label{forins}
			\Left$\leftarrow$ \FindCompress{$Im[i,j-1]$}\;
			\Up$\leftarrow$ \FindCompress{$Im[i-1,]$}\;
			\This$\leftarrow$ \FindCompress{$Im[i,j]$}\;
			\If(\tcp*[h]{O(\Left,\This)==1}){\Left compatible with \This}{\label{lt}						\lIf{\Left $<$ \This}{\Union{\Left,\This}}
				\lElse{\Union{\This,\Left}}
			}
			\If(\tcp*[f]{O(\Up,\This)==1}){\Up compatible with \This}{\label{ut}
				\lIf{\Up $<$ \This}{\Union{\Up,\This}}
				\tcp{\This is put under \Up to keep tree as flat as possible}\label{cmt}					\lElse{\Union{\This,\Up}}\tcp*[h]{\This linked to \Up}\label{lelse}}
				}
			\lForEach{element $e$ of the line $i$}{\FindCompress{p}}
		}
		\caption{Bucket Sort}
\end{algorithm}






\end{multicols}














\end{document}
