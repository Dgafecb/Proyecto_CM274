\documentclass[11pt,a4paper,spanish]{article}
\usepackage[latin1]{inputenc}
\usepackage{amsmath}
\usepackage{amsfonts}
\usepackage{amssymb}
\usepackage{graphicx}
\usepackage{multicol}
%Ignoren estas 2 lineas, son solo para definir la funcion piso para el pseudocodigo
\usepackage{mathtools}
\DeclarePairedDelimiter\floor{\lfloor}{\rfloor}

\usepackage[boxruled, linesnumbered]{algorithm2e}
\usepackage[left=1.65cm, right=1.65cm, top=1.78cm, bottom=1.78cm]{geometry}

\author{Bollschweiler Vargas Ian Nicolas\\
			Cabrejos Rafael Erika Epifania\\
			Hidalgo Esquivel Jared Miguel\\
		 	Del Rosario Sosa Joshua Jean Paul	}
\title{\bf TIEMPO DE VALOR ESPERADO EN ALGORITMOS DE ORDENACI\'ON}
\date{}
\pagestyle{empty}
\begin{document}
	\maketitle
\begin{center}
		\textbf{Resumen}\\
En este informe plantearemos la eficiencia de algunos algoritmos de ordenamiento, es decir, calcularemos el valor esperado de tiempo de ejecuci\'on para una entrada aleatoria de tama\~no n. Este tema es importante ya que para diversas aplicaciones (como la industria de los videojuegos) los algoritmos de ordenamiento son muy utilizados y algunos aspectos (como la popularidad del juego) se ver\'an afectados por la eficiencia del algoritmo (un juego que se "cuelga" mucho podr\'ia desanimar a sus seguidores).   

\end{center}
\begin{multicols}{2}
\section{\normalsize INTRODUCCI\'ON}
Los algoritmos de ordenamiento, como bien indica su nombre, nos permiten ordenar, es decir
darle a cada objeto(o dato) una posici�n determinada seg�n cierto criterio espec�fico.\\
En este caso, nos enfocaremos en ordenar un conjunto datos, datos como:  Vectores o matrices.\\
En este informe nos efocaremos en el estudio de algunos de los algoritmos de ordenamiento m�s utilizados, analizando la
cantidad de comparaciones que se dan, el tiempo que toma cada comparaci�n y la extensi�n de su c�digo, para cada algoritmo analizado.\\
Este informe nos permitir� conocer y comprender m�s a fondo cada uno de los m�todos  
de ordenamiento analizados, desde los m�s simples hasta los m�s complejos. Se realizaran comparaciones
en tiempo de ejecucion, pre-requisitos de cada algoritmo, funcionalidad, alcance, entre otros.
Gracias a los conocimientos obtenidos en clase se determinar� una eficiencia (porcentual) de cada algoritmo,
se har� uso de criterios como la media y la mediana para este objetivo.
a su vez, daremos una vista gr�fica de los resultados obtenidos usando lenguaje R para graficar los tiempos y otros datos obtenidos de los algoritmos de ordenamineto.
\section{{\normalsize ESTADO DEL ARTE}}
\begin{itemize}
\item \textbf{Performance analysis of Sorting Algorithms:} \\En esta tesis el autor realiza una vista general sobre los algoritmos de ordenaci\'on y un an\'alisis de sus respectivos rendimientos .
\item \textbf{Run-Time Analysis for sorting algorithms:}\\Este art\'iculo trata sobre la evaluaci\'on de 3 algoritmos de ordenaci\'on en 3 distintos lenguajes de programaci\'on para aproximar su complejidad .

\end{itemize}

\section{{\normalsize DISE�O DEL EXPERIMENTO}}
Para encontrar el tiempo de valor esperado, implementaremos 5 algoritmos de ordenamiento, siendo estos: Merge sort, Bubble sort, Quick sort, Insertion sort y Bucket sort. Estos ser�n implementados en lenguaje R donde ordenar�n un arreglo de n�meros aleatorios de n elementos. Este proceso se realizar� 1000 veces y se tomar�n los tiempos de ejecuci�n de cada algoritmo para hallar una distribuci\'on en intervalos para cierta cantidad n de elementos. Despu�s, se proceder� a encontrar la esperanza del tiempo de ejecuci\'on a trav\'es de un c\'alculo matem\'atico de los datos para distintos valores de n. Finalmente, se proceder\'a a realizar una gr\'afica con los tiempos esperados de cada algoritmo para diferentes tama\~nos de entrada n.\newline Los pseudoc�digos de los algoritmos a utilizar son:


\begin{algorithm}[H]	
	\SetKwData{A}{A}
    \SetKwFunction{MergeSort}{MergeSort}\SetKwFunction{Merge}{Merge}
    \SetKwInOut{Input}{Entrada}\SetKwInOut{Output}{Salida}
    \Input{Array A,p,r}
    \Output{Array A ordenado}
    \BlankLine
    \If(){p $<$ r}{
        \emph{ q = $\floor*{\frac{(p+r)}{2}}$}\;
        \MergeSort{A,p,q}\;
        \MergeSort{A,q+1,r}\;
        \Merge{A,p,q,r}\;
        \tcp*[h]{Merge junta los arreglos A[p..q] y A[q+1..r] que se encuentran ordenados}
    }
    	\caption{Merge Sort}
\end{algorithm}


\begin{algorithm}[H]	
    \SetKwInOut{Input}{Entrada}\SetKwInOut{Output}{Salida}
    \SetKwFunction{Intercambiar}{Intercambiar}
    \Input{Array A con n elementos}
    \Output{Array A ordenado}
    \BlankLine
    \For{$i\leftarrow 1$ \KwTo $n$}{
		\For{$j\leftarrow 1$ \KwTo $n$}{        
        	\If(){A[j] $>$ A[j+1]}{
        		\emph{\Intercambiar{A[j], A[j+1]}}\;
			}        
        }
    }
    	\caption{Bubble Sort}
\end{algorithm}

\begin{algorithm}[H]	
	\SetKwFunction{Particion}{Partici\'on}
	\SetKwFunction{QuickSort}{QuickSort}
    \SetKwInOut{Input}{Entrada}\SetKwInOut{Output}{Salida}
    \Input{Array A,p,r}
    \Output{Array A ordenado}
    \BlankLine
    \If(){p$<$r}{
    	\emph{q $=$ \Particion{A,p,r}}\;
    	\tcp*[h]{La funci�n partici�n reordena el arreglo A[p..r]}
    	\emph{\QuickSort{A,p,q-1}}\;
    	\emph{\QuickSort{A,q+1,r}}\;    
    }    
    
    
   	\caption{Quick Sort}
\end{algorithm}

\begin{algorithm}[H]	
	
	\SetKwInOut{Input}{Entrada}\SetKwInOut{Output}{Salida}
    \Input{Array A con n elementos}
    \Output{Array A ordenado}
    \For{$j\leftarrow 2$ \KwTo n}{
		\emph{llave $=$ $A[j]$}\;
		\emph{$i = j-1$}\;
		\While(){i$>$0 y $A[i] > llave$}{
			\emph{$A[i+1] = A[i]$}\;
			\emph{$i = i-1$}\;
		}
		\emph{$A[i+1] = llave$}\;	
	}
    	\caption{Insertion Sort}
\end{algorithm}

\begin{algorithm}[H]	
	
	\SetKwInOut{Input}{Entrada}\SetKwInOut{Output}{Salida}
    \Input{Array A con n elementos}
    \Output{Array A ordenado}
    \BlankLine
    \emph{Sea $B[0..n-1]$ un nuevo array}\;
    \For{$j\leftarrow 0$ \KwTo $n-1$}{
    	\emph{Hacer B[i] una lista vacia}\;
    }
    \For{$i\leftarrow 1$ \KwTo $n$}{
		\emph{Insertar $A[i]$ en la lista $B[\floor*{nA[i]}]$}\;    
    }
    \For{$i\leftarrow 0$ \KwTo $n-1$}{
		\emph{Ordenar la lista $B[i]$ usando el Insertion Sort}\;    
    }
	\emph{Concatenar las listas $B[0],B[1]...B[n-1]$ en orden}\;       
    \caption{Bucket Sort}
\end{algorithm}




\end{multicols}
\end{document}
