\documentclass[11pt,a4paper,spanish]{article}
\usepackage[latin1]{inputenc}
\usepackage{amsmath}
\usepackage{amsfonts}
\usepackage{amssymb}
\usepackage{graphicx}
\usepackage{multicol}
\usepackage[left=1.65cm, right=1.65cm, top=1.78cm, bottom=1.78cm]{geometry}

\author{Bollschweiler Vargas Ian Nicolas\\
			Cabrejos Rafael Erika Epifania\\
			Hidalgo Esquivel Jared Miguel\\
		 	Del Rosario Sosa Joshua Jean Paul	}
\title{\bf TIEMPO DE VALOR ESPERADO EN ALGORITMOS DE ORDENACI\'ON}
\date{}
\pagestyle{empty}
\begin{document}
	\maketitle
\begin{center}
		\textbf{Resumen}
	\begin{abastract}
	%contenido del resumen 
	\end{abstract}
\end{center}
\begin{multicols}{2}
\section{\normalsize INTRODUCCI\'ON}}

\section{{\normalsize ESTADO DEL ARTE}}
\begin{itemize}
\item \textbf{Shell Sort :} es principalmente una variaci�n de Insertion Sort .En la ordenaci�n por inserci�n, movemos los elementos solo una posici�n m�s adelante. Cuando un elemento tiene que moverse mucho m�s adelante, muchos movimientos est�n involucrados.
Durante la ejecuci\'on de este algoritmo, los n\'umeros de la lista se van casi-ordenando y finalmente, el \'ultimo paso o funci\'on de este algoritmo es un simple metodo por inserci\'on que, al estar casi-ordenados los n\'umeros, es m�s eficiente.
\item \textbf{Merge Sort:} el algoritmo de Ordenamiento por mezcla (Merge sort en ingl�s) es un algoritmo de ordenaci�n externo estable basado en la t�cnica divide y vencer�s. Es de complejidad O($n \log n$).Fue desarrollado en 1945 por John Von Neumann.
A grandes rasgos, el algoritmo consiste en dividir en dos partes iguales el vector a ordenar, ordenar por separado cada una de las partes, y luego mezclar ambas partes, manteniendo el orden, en un solo vector ordenado.
\item\textbf{Quick Sort:} el ordenamiento r�pido (quick sort en ingl�s) es un algoritmo basado en la t�cnica de divide y vencer�s, que permite, en promedio, ordenar n elementos en un tiempo proporcional a O($n \log n$). Esta es la t�cnica de ordenamiento m�s r�pida conocida. Fue desarrollada por C. Antony R. Hoare en 1960. El algoritmo original es recursivo, pero se utilizan versiones iterativas para mejorar su rendimiento .


\end{itemize}
\section{{\normalsize DISE�O DEL EXPERIMENTO}}

\end{multicols}
\end{document}